\documentclass{article}
\usepackage{listings}
\usepackage{graphicx}

\lstset{ %
language=Java,                  % the language of the code
basicstyle=\footnotesize,       % the size of the fonts that are used for the code
numbers=left,                   % where to put the line-numbers
numberstyle=\footnotesize,      % the size of the fonts that are used for the line-numbers
stepnumber=1,                   % the step between two line-numbers. If it's 1, each line 
                                % will be numbered
numbersep=5pt,                  % how far the line-numbers are from the code
showspaces=false,               % show spaces adding particular underscores
showstringspaces=false,         % underline spaces within strings
showtabs=false,                 % show tabs within strings adding particular underscores
frame=single,                   % adds a frame around the code
tabsize=2,                      % sets default tabsize to 2 spaces
captionpos=t,                   % sets the caption-position to bottom
breaklines=true,                % sets automatic line breaking
breakatwhitespace=false,        % sets if automatic breaks should only happen at whitespace
title=\lstname,                 % show the filename of files included with \lstinputlisting;
}

\begin{document}

\setlength{\parskip}{\medskipamount}
\setlength{\parindent}{0pt}

\title{Extending SPARQL With Remote OWL Reasoning}
\author{Luke Slater (lus11@aber.ac.uk)}
\date{February 2014}

\maketitle

\pagebreak

\section{Introduction}

My project is concerned with interlinking two semantic web technologies,
allowing reasoning to take place in a remote system from a local endpoint and
allowing these results to be utilised by the initial query. This is undertaken
with a view to increase the interoperability of semantic web technologies, and
increase the power and usefulness of the federation they work in.

Specifically, the project will be the creation of an addition to
SPARQL\cite{sparql} (SPARQL Protocol And RDF Query Language) to allow remote
OWL\cite{owlprimer} reasoning to be performed on OWLlink endpoints and be included in the
SPARQL result-set.

There is a similar construct already included in the SPARQL language called
\emph{SERVICE}, which is unfortunately not suitable for use in this project
because it won't support the syntax of the OWL language within it (however, it
might be worth contacting the developers of the SPARQL specification asking if a
generalisation would be prudent).

So, I will develop a similar addition to the language, such as an \emph{OWL}
block, which will allow the language to execute a given OWL query at a given
OWLlink endpoint (along with the possibility of a flag depending on the type of
SPARQL query), and then retrieve these results and use them in the SPARQL
result-set.

Furthermore, I will endeavour to create a working implementation of this
addition - which will allow a demonstration of the system. This will require
some research into the various technologies currently available.

Upon finishing the research project, the findings will be presented in the form
of a scientific paper presenting the specification of the addition to the
language, its impact and its uses.

\section{State of The Art in The Semantic Web}

\subsection{The Semantic Web}

The Semantic Web is a set of technologies and methodologies suitable for
generalised data expression, transmission and processing over the World Wide
Web. Before this, up until the Semantic Web's philosophical beginnings in
2001, the World Wide Web had seen success as a platform for the
sharing resources intended for reading and interaction by humans.

This meant that most of the data existed in the format of forward-facing 
documents accessible via URLs, content comprising of arbitrary natural language
along with human-intended formatting and navigation techniques.

The downside of this, is that without using advanced natural language 
processing techniques, it's very difficult to then have software process and
intercommunicate this data in a generalised and useful manner.

Therefore, the philosophy of the semantic web acknowledges this need for a
reconciliation of certain data in a format which software can deal in. The
Semantic Web\cite{semweb} describes the vision of a futuristic world of home
automation and service intercommunication serving the human lifestyle, backed by
these data formats which allow software to easily traverse the totality of
available data and process it in a meaningful manner. 

\subsection{RDF}

One of the primary technologies used to represent this machine-workable data on
the web is the Resource Description Framework (RDF), which are a set of
specifications which describe a general methodology for conceptualising and
modelling data, intended for solutions in which the data needs to be processed
by applications.

It forms a graph of statements representing metadata about resources on the
World Wide Web in the form of simple statements in the terms of simple
properties and property values. Each RDF statement is formed of a Subject, a
Predicate and an Object:

\begin{description}
    \item[Subject] The object being described.
    \item[Predicate] The definition of the property of the subject being
    defined - usually given as a URI reference or URIRef.
    \item[Object] This is the value assigned to the given
    predicate-defined property of the subject.
\end{description}

For example, in the simplest terms a statement which defines the name of the author of
this document might take the form:

http://users.aber.ac.uk/lus11/dissertation has an author whose value is Luke
Slater.

In which the subject is \emph{http://users.aber.ac.uk/lus11/dissertation}, the
predicate is \emph{author} and the object is \emph{Luke Slater}.

RDF information is commonly stored, transmitted and worked with using the
RDF/XML format, which is an expression of the graph data structures in the
eXtensible Markup Language (XML) - a language designed to store and encode
documents (as opposed to HTML, which is for displaying data). Alternatively, and
in the remainder of this document, the Turtle format is used to communicate RDF
in a format which is both computer and human readable.

Usually, URIs are used to identify objects, subjects and predicates because they
allow them to be fully-fledged resources and in using already-defined
vocabularies one can avoid using multiple strings to refer to the same thing.

So, a full example

\subsection{SPARQL}

SPARQL (SPARQL Protocol And RDF Query Language) is a query language similar to
SQL which allows the querying and manipulation of RDF (and similar) datastores.
This is a human-interactable language, which means that while it has a
non-freeform syntax it is designed to be used by humans (as well as computers)
to interact with data on the semantic web.

SPARQL queries usually consist of the following:

\begin{description}
    \item[PREFIX] A list of prefix declarations, which can be used to shorten
    and simplify URIs in the remainder of the query.
    \item[FROM] Describes the dataset to query, usually a URI pointing at an RDF
    dataset.
    \item[SELECT] Describes the data to include in the resultset.
    \item[WHERE] Describes the data to query for in the dataset.
    \item[MODIFIERS] These allow you to apply modifiers to the resultset, such
    as a limitation on the total results or a pattern for ordering.
\end{description}

Less commonly there are also CONSTRUCT, which allows you to pull full triples of
data from a store.

SPARQL also has the provision to query data from non-RDF databases such as Redis
and OWL ontologies by transforming and interacting with the data as if it were
in an RDF syntax. This is useful for data integration, as it means data can be
used over multiple formats. 

Another important quality of SPARQL is that it's federated, data may be loaded
remotely over the web by providing an IRI to the datastore. Also, the language
includes a 'SERVICE' keyword, which allows a query to be sent to a remote SPARQL
endpoint and be executed and results be returned and included in the resultset.

A SPARQL endpoint is a simple interface which accepts a HTTP query including a
query parameter, then runs the query and will return the results in one of 
several machine-readable formats (usually JSON or XML). This means a wide array
of applications can easily make use of the data SPARQL deals in - one example
being Virtuoso. It also means that simple online interfaces can be developed for 
working with SPARQL. These qualities mean that SPARQL makes RDF very universally
available for use.

\subsection{OWL Ontologies}

* Describing ontologies
* Describe OWL

\section{Problem Statement}

\subsection{Data Integration}

The issue with the Semantic Web as it currently exists is that while there is a
lot of data stored in machine-processable formats, there is a lot of information 
existing in different data models (RDF and OWL ontologies, in respects to this project). 
This means it becomes somewhat difficult to integrate data between said models 
if they are representing similar data in different ways.

There is already some provision for this, as explained in the previous section - 
SPARQL does actually allow data to be queried from an OWL ontology, loading a full 
list of objects in an ontology designated by an IRI and then being able to query 
it in an RDF expression. However, while this is useful in some cases, it misses 
two major advantages of OWL ontologies: the performance of semantic reasoning
upon the dataset, and the ability to query the data in a native manner (using 
Manchester OWL Syntax).

Therefore the problem in this case is developing a manner by which to integrate
data stored in RDF and OWL formats without compromising the endemic qualities of 
each model through doing so.

* Why it's useful

\subsection{Human-interactable OWL Querying Over The Web}

As described in the State of The Semantic Web, there are two categories of
languages used to represent and interact with the data: human-interactable and
machine-interactable. These generally exist as the data storage formats (such as
RDF or OWL) and the query languages (such as SPARQL or Manchester OWL Syntax)
which can be used by humans to manipulate and integrate the datastores.

An issue with the Semantic Web is also that of exposing these datasets to use by
computers and machines over the Web i.e. data avilability and integration over
the web.

For both RDF and OWL datastores it is currently easy to interact with the data
over the web in a machine-processable manner - primarily through codebases
utilising XML parsers or through bespoke libraries and server software such as 
OWLAPI and OWLLink.

For human-querying, the area is highly-developed for RDF and similar datastores, 
with SPARQL allowing users to write queries to manipulate data both locally and
through web-based endpoints. The language also supports federation through the
SERVICE keyword, which allows the user to send further data queries to remote
SPARQL endpoints and thereby further datasets. This is incredible useful for
making useful data easily available and integratable to those working with it.

However, for OWL ontologies the situation is somewhat different,
human-interaction with ontologies is generally done through desktop applications
such as Protégé, which allows users to load, edit, reason and query ontologies 
using the Manchester OWL Syntax.

However, there is currently no way for users to send simple Manchester OWL
Syntax queries to remote reasoners and retrieve relevant classes for use. 
Therefore, another problem is providing a web-based solution for human-input OWL
queries over multiple ontologies through Web. 

\section{Design and Implementation}

My solution to the given problems are to develop an extension to the SPARQL
query language which allows a user to send a Manchester OWL Syntax query to a
remote OWL reasoner endpoint, which will return relevant objects which are
included in the SPARQL resultset. This will require both development of the
language extension and the queriable OWL endpoint.

\subsection{SPARQL Extension}

As described in the State of The Art, there is already a SERVICE keyword
included in the SPARQL language which provides a good model for sending a text
query to a remote endpoint. Currently, it only supports remote SPARQL endpoints,
and it was decided not to extend this construct to include support for an OWL
endpoint as this would mean any resulting implementation would break the terms
of the SPARQL specification. However, it may be suggested to the W3C in future
to combine these constructs as a matter of efficiency. 

Therefore, my software will add an 'OWL' block to the SPARQL language, which
uses a similar syntactic pattern to the SERVICE keyword.

\begin{lstlisting}
OWL <http://realispicio.us:9090> {
    Pizza and hasTopping some FishTopping
}
\end{lstlisting}

There are three properties to the above query:

\begin{description}
    \item[OWL] The OWL keyword will designate the beginning of an OWL block
    denoting the beginning of the description for a query to be sent to a remote
    OWL endpoint.
    \item[IRI] The IRI, contained between angled brackets following the OWL
    keyword, will define the URI of the remote OWL endpoint.
    \item[Query] Following, between the curly brackets, will be a Manchester OWL
    Syntax query referring to relevant objects to be returned to the query.
\end{description}

\section{Testing}

\section{Discussion}

\section{Project Deliverables}

\begin{itemize}
  \item \emph{Specification} - This will involve a description of the language additions 
  and their syntax in a 'specification'-type format. 
  \item \emph{Implementation} - An implementation of the above addition in the
  form of an addition to a piece of software which iplements a SPARQL server to
  allow for the running of the queries detailed above.
  \item \emph{Working Demonstration} - In the form of a working SPARQL endpoint
  and an OWLlink endpoint, with the ability for the former to run queries on the
  latter through the previously developed language extension.
  \item \emph{Tests} - A couple of tests (either manual or automatic) which
  prove the extensions as working.
  \item \emph{Progress Report} - This involves a mid-period report which covers
  an overview of the technologies learned, and details on the specification and 
  implementation of the extension to SPARQL.
  \item \emph{Paper} - Scientific paper describing the results of the research
  project.
  \item \emph{Final Report} - Report covering the entire project, including
  a full project description, background, impact and significance, 
  implementation process, documentation and evaluation of final progress.
\end{itemize}

\section{Conclusion}

* Future
* Directions for the project itself
* What can build on it

\begin{thebibliography}{9}

\bibitem{semweb}
  Tim Berners-Lee et al,
  \emph{The Semantic Web}.
  Feature Article, Scientific American
  May 2001.
    \\  
    \emph{This article is definitive in describing the initial idea and goals for the
    'Semantic Web,' by the project's founders. It provides the 'vision' for the
    semantic web and is therefore an underlying background for all of the
    technologies used in this project.}

\bibitem{rdfprimer}
 Tim Berners-Lee et al,
 \emph{RDF Primer}.
 W3C,
 2004.
   \\ 
   \emph{This W3C 'primer' document provides a detailed description of the RDF
   datastore - its uses, significance and implementation details. Important to
   gain an understanding of the data format which will be one of the two
   technologies being connected.}

\bibitem{sparql}
  Eric Prud'hommeaux, Andy Seaborne,
  \emph{SPARQL Query Language for RDF}.
  W3C,
  January 2008.
    \\ 
    \emph{This specification describes the SPARQL query language, used for
    interaction with RDF datastores. Good for a background and details of 
    the technology I will be extending.}

\bibitem{desclogic}
 Franz Baader, Deborah L. McGuinness, Daniele Nardi, Peter F. Patel-Schneider,
 \emph{The Description Logic Handbook - Theory, Implementation and Actions}.
 Cambridge University Press,
 March 2003.
   \\ 
   \emph{This book provides the background for description logic, which is the
   grounding for the OWL language.}

\bibitem{owlprimer}
  Pascal Hitzler et al,
  \emph{OWL 2 Web Ontology Language Primer (Second Edition)}.
  W3C,
  December 2012.
  \\ 
  \emph{This is a primer on the language which is going to be implemented in
  SPARQL. Good for understanding the logic and usability.}

\end{thebibliography}

\end{document}
