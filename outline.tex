\documentclass{article}
\usepackage{listings}
\usepackage{graphicx}

\lstset{ %
language=Java,                  % the language of the code
basicstyle=\footnotesize,       % the size of the fonts that are used for the code
numbers=left,                   % where to put the line-numbers
numberstyle=\footnotesize,      % the size of the fonts that are used for the line-numbers
stepnumber=1,                   % the step between two line-numbers. If it's 1, each line 
                                % will be numbered
numbersep=5pt,                  % how far the line-numbers are from the code
showspaces=false,               % show spaces adding particular underscores
showstringspaces=false,         % underline spaces within strings
showtabs=false,                 % show tabs within strings adding particular underscores
frame=single,                   % adds a frame around the code
tabsize=2,                      % sets default tabsize to 2 spaces
captionpos=t,                   % sets the caption-position to bottom
breaklines=true,                % sets automatic line breaking
breakatwhitespace=false,        % sets if automatic breaks should only happen at whitespace
title=\lstname,                 % show the filename of files included with \lstinputlisting;
}

\begin{document}

\setlength{\parskip}{\medskipamount}
\setlength{\parindent}{0pt}

\title{Outline Project Specification}
\author{Luke Slater (lus11@aber.ac.uk)}
\date{February 2014}

\maketitle

\pagebreak

\section{Project Description}

My project is concerned with interlinking two semantic web technologies,
allowing reasoning to take place in a remote system from a local endpoint and
allowing these results to be utilised by the initial query. This is undertaken
with a view to increase the interoperability of semantic web technologies, and
increase the power and usefulness of the federation they work in.

Specifically, the project will be the creation of an addition to
SPARQL\cite{sparql} (SPARQL Protocol And RDF Query Language) to allow remote
OWL\cite{owlprimer} reasoning to be performed on OWLlink endpoints and be included in the
SPARQL result-set.

There is a similar construct already included in the SPARQL language called
\emph{SERVICE}, which is unfortunately not suitable for use in this project
because it won't support the syntax of the OWL language within it (however, it
might be worth contacting the developers of the SPARQL specification asking if a
generalisation would be prudent).

So, I will develop a similar addition to the language, such as an \emph{OWL}
block, which will allow the language to execute a given OWL query at a given
OWLlink endpoint (along with the possibility of a flag depending on the type of
SPARQL query), and then retrieve these results and use them in the SPARQL
result-set.

Furthermore, I will endeavour to create a working implementation of this
addition - which will allow a demonstration of the system. This will require
some research into the various technologies currently available.

Upon finishing the research project, the findings will be presented in the form
of a scientific paper presenting the specification of the addition to the
language, its impact and its uses.

\section{Proposed Tasks}

\subsection{Research}

Being a research project, most of the work involved will be research-orientated
in learning the relevant technologies and their inter-workings. Starting with
background information on the Semantic Web\cite{semweb} and then onto the specifications and
workings of the technologies designed to implement this vision. After that,
specifics will need to be learned about certain implementations of said
technologies required to implement a working model of the proposed system.

The main technologies involved which will need to be learned are
RDF\cite{rdfprimer}, SPARQL, Turtle, OWL and OWLlink. Gaining an understanding
of OWL will also require some background in Description Logic\cite{desclogic}.

\subsection{Implementation}

The implementation will centre around utilising the technologies researched in
previously to implement the target functionality. I will have to pick an
implementation of SPARQL and on top of that develop the extension which allows it to
make remote OWL requests on an OWLlink server. There are a few implementations
existing, but the major OWL API looks like a good choice given its widespread
adoptance.

Testing the implementation and creating the working demonstration will involve
the deployment of a SPARQL endpoint which has the ability to make remote OWL
requests. For this I will also host an OWLlink server which will be able to
return the results.

Either automatic testing or manual testing may take place depending on the
provisions for this made in the various softwares.

\section{Project Deliverables}

\begin{itemize}
  \item \emph{Specification} - This will involve a description of the language additions 
  and their syntax in a 'specification'-type format. 
  \item \emph{Implementation} - An implementation of the above addition in the
  form of an addition to a piece of software which iplements a SPARQL server to
  allow for the running of the queries detailed above.
  \item \emph{Working Demonstration} - In the form of a working SPARQL endpoint
  and an OWLlink endpoint, with the ability for the former to run queries on the
  latter through the previously developed language extension.
  \item \emph{Tests} - A couple of tests (either manual or automatic) which
  prove the extensions as working.
  \item \emph{Progress Report} - This involves a mid-period report which covers
  an overview of the technologies learned, and details on the specification and 
  implementation of the extension to SPARQL.
  \item \emph{Paper} - Scientific paper describing the results of the research
  project.
  \item \emph{Final Report} - Report covering the entire project, including
  a full project description, background, impact and significance, 
  implementation process, documentation and evaluation of final progress.
\end{itemize}

\begin{thebibliography}{9}

\bibitem{semweb}
  Tim Berners-Lee et al,
  \emph{The Semantic Web}.
  Feature Article, Scientific American
  May 2001.
    \\  
    \emph{This article is definitive in describing the initial idea and goals for the
    'Semantic Web,' by the project's founders. It provides the 'vision' for the
    semantic web and is therefore an underlying background for all of the
    technologies used in this project.}

\bibitem{rdfprimer}
 Tim Berners-Lee et al,
 \emph{RDF Primer}.
 W3C,
 2004.
   \\ 
   \emph{This W3C 'primer' document provides a detailed description of the RDF
   datastore - its uses, significance and implementation details. Important to
   gain an understanding of the data format which will be one of the two
   technologies being connected.}

\bibitem{sparql}
  Eric Prud'hommeaux, Andy Seaborne,
  \emph{SPARQL Query Language for RDF}.
  W3C,
  January 2008.
    \\ 
    \emph{This specification describes the SPARQL query language, used for
    interaction with RDF datastores. Good for a background and details of 
    the technology I will be extending.}

\bibitem{desclogic}
 Franz Baader, Deborah L. McGuinness, Daniele Nardi, Peter F. Patel-Schneider,
 \emph{The Description Logic Handbook - Theory, Implementation and Actions}.
 Cambridge University Press,
 March 2003.
   \\ 
   \emph{This book provides the background for description logic, which is the
   grounding for the OWL language.}

\bibitem{owlprimer}
  Pascal Hitzler et al,
  \emph{OWL 2 Web Ontology Language Primer (Second Edition)}.
  W3C,
  December 2012.
  \\ 
  \emph{This is a primer on the language which is going to be implemented in
  SPARQL. Good for understanding the logic and usability.}

\end{thebibliography}

\end{document}
